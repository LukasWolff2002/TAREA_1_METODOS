\section{Introducción}

La tarea consiste en realizar la cubicación y programación de algunas obras involucradas en la construcción del Aeropuerto y evaluar los plazos para cada actividad indicada. Específicamente, se trata de un proyecto para un paso bajo nivel en una nueva vía de circulación denominada Diego Barros Ortiz. Las partidas incluyen actividades relacionadas con excavaciones, muros de contención, pavimentación y la construcción de cajones de hormigón armado. Esta tarea no solo implica un cálculo preciso de volúmenes, dimensiones y tiempos, sino que también tiene un impacto real en la planificación y ejecución de obras civiles, dado que una adecuada cubicación y evaluación de plazos son esenciales para cumplir con los requisitos técnicos, legales y financieros de proyectos de infraestructura. 
\\\\
El proceso de cubicación y planificación es crucial para determinar la viabilidad y el éxito de una obra civil. En proyectos reales, errores en estos cálculos pueden generar sobrecostos, retrasos y complicaciones técnicas que afectan tanto a los contratistas como a los usuarios finales de la infraestructura. Además, la correcta programación de actividades con un orden de precedencia lógico, apoyado en datos confiables y herramientas como MS Project, permite optimizar los recursos y garantizar una ejecución eficiente. Tal como se plantea más adelante, estos fueron los objetivos principales que se abordaron en la realización de este proyecto.
