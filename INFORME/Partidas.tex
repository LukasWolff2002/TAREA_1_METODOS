\section{Descripción del Proyecto}

El proyecto consite en la ampliación del Aeropuerto construyendo 
la extensión de la calle Diego Barros Ortiz y su respectivo paso 
bajo nivel, el cual cruza inferiormente a la pista de rodaje Mike.
Las Partidas incluidas en el proyecto son las siguientes:

\begin{itemize}
    \item \textbf{VIALIDAD}
    \begin{itemize}
        \item \textbf{Preparación área de trabajo}
        \begin{itemize}
            \item Remoción de Pavimento de Hormigón ($m^3$)
            \item Remoción de Pavimento Asfáltico ($m^3$)
        \end{itemize}
        \item \textbf{Movimiento de Tierra}
        \begin{itemize}
            \item Excavación en Terreno de Cualquier Naturaleza (TCN) ($m^3$) 
            \item Formación y COnstrucción de Terraplenes ($m^3$)
        \end{itemize}
        \item \textbf{Capas Granulares}
        \begin{itemize}
            \item Sub Base Granular CBR 40\% ($m^3$)
            \item Base Granular CBR 80\% ($m^3$)
            \item Carpeta Granular de Rodadura CBR 60\% ($m^3$)
        \end{itemize}
        \item \textbf{Pavimentos}
        \begin{itemize}
            \item Imprimación ($m^2$)
            \item Riego de Liga ($m^2$)
            \item Concreto Asfáltico, Capa de Rodadura ($m^3$)
            \item Concreto Asfáltico, Capa Intermedia ($m^3$)
            \item Aceras de Hormigón ($m^2$)
            \item Soleras Tipo A ($m$)
        \end{itemize}
        \item \textbf{Señalética y Demarcación}
        \begin{itemize}
            \item Señal Vertical Lateral Tipo 2 ($UN$)
            \item Señal Vertical Lateral Tipo 3 ($UN$)
            \item Líneas de Eje Segmentadas ($m$)
            \item Línea Continua ($m$)
            \item Líneas, Achurados, Símbolos y Leyendas ($m^2$)
        \end{itemize}
    \end{itemize}
    \item \textbf{ESTRUCTURAS}
    \begin{itemize}
        \item Excavación a Máquina en Puente y Estructuras ($m^3$)
        \item Relleno Estructural Permeable ($m^3$)
        \item Impermeabilización de Muros y Estribos ($m^2$)
        \item Hormigón G-5 ($m^3$)
        \item Hormigón G-30/SM ($m^3$)
        \item Acero para Armaduras A63-42H ($kg$)
        \item Moldajes para Infraestructura ($m^2$)
        \item Losa de Acceso ($m^3$)
        \item Barbacanas de desagüe ($UN$)
    \end{itemize}
\end{itemize}
\vspace{1cm}
A cada una de estas se le asignó un rendimiento por unidad de medida para realizar la programación en Project.
\\
\subsection{Aclaraciones y Supuestos}

\begin{itemize}
    \item La cubicación de cada etapa se hizo exclusivamente a la extensión de la calle DBO.
    \item Se supuso que toda la extensión de DBO tiene igual área transversal superficial para la excavación en terreno de cualquier naturaleza (TCN).
    \item Para la excavación a máquina en puentes y estructuras se tomó en cuenta el emplantillado de las fundaciones y la forma trapezoidal de estas.
    \item El detalle técnico de los estribos no aparecen en los archivos DWG, pero igual se contaron y tomaron en cuenta para la cantidad de acero.
    \item Se estimó el rendimiento de algunas partidas dada la experiencia en obras durante la práctica de los estudiantes.
\end{itemize}

