\section{Programación de Partidas}

A continuación, se detallará y analizará la elección de plazos y programación de las obras a ejecutar.

\subsection{Vialidad y Estructuras}

Las obras se inician el dia lunes 30 de septiembre con la demolición del
pavimento existente. Posteriormente, se procede a excavar el terreno natural
para preparar el área de trabajo, seguido de la excavación y formación de
terraplenes junto con la excavación a máquina para puentes e infraestructura
necesaria para el paso bajo nivel. 
\\\\
Seguido de la preparación del terreno, en la parte de estructuras, se coloca el hormigón G-5 
para la base de las fundaciones de muros y cajones, se prosigue con el armado de moldajes y 
encofrados y se comienza la preparación de armadura estructural. Teniendo listos los moldajes, 
se vierte el hormigón G-30/SM para muros, losas de acceso y cajones de bajo nivel. Tiempo después, 
una vez que las estructuras tengan su resistencia óptima, se procede a la impermeabilización de las 
mismas y luego se aplica el relleno permeable. Para finalizar, se cubre con tierra, se compacta 
y se termina de pavimentar esa sección con el mismo proceso mencionado en la parte de vialidad.
\\\\
Por otro lado, en la sección de vialidad, una vez terminadas las estructuras, se procede
a preparar la sub rasante, añadir las sub bases y bases granulares para la recepción de 
la carpeta asfáltica y sus respectivas compactaciones. Luego, se imprima la base, se
aplica el riego de liga y se procede a colocar la capa intermedia de asfalto y 
la carpeta asfáltica de rodadura. Posteriormente, para ayudar como barrera de soporte del 
hormigonado para vereda, se hace la instalación de las soleras tipo A. Para finalizar, se vierte el
hormigón para la construcción de la acera y cuando finalice su curado, se procede a la 
instalación de señaléticas y demarcación requeridas.
\\\\
Las partidas se realizaron en el orden mencionado dado que es lógico y eficiente. Además, sigue un
orden de ejecución que permite que las partidas se realicen de manera continua y sin interrupciones
dentro de lo posible, para que el plazo total de la obra sea el menor posible.

\subsection{Rendimientos por partida}
