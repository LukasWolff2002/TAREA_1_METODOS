\section{Programación de Partidas}

A continuación, se detallará y analizará la elección de plazos y programación de las obras a ejecutar.

\subsection{Vialidad y Estructuras}

Las obras se inician el dia lunes 30 de septiembre con la demolición del
pavimento existente. Posteriormente, se procede a excavar el terreno natural
para preparar el área de trabajo, seguido de la excavación y formación de
terraplenes junto con la excavación a máquina para puentes e infraestructura
necesaria para el paso bajo nivel. 
\\\\
Seguido de la preparación del terreno, en la parte de estructuras, se coloca el hormigón G-5 
para la base de las fundaciones de muros y cajones, se prosigue con el armado de moldajes y 
encofrados y se comienza la preparación de armadura estructural. Teniendo listos los moldajes, 
se vierte el hormigón G-30/SM para muros, losas de acceso y cajones de bajo nivel. Tiempo después, 
una vez que las estructuras tengan su resistencia óptima, se procede a la impermeabilización de las 
mismas y luego se aplica el relleno permeable. Para finalizar, se cubre con tierra, se compacta 
y se termina de pavimentar esa sección con el mismo proceso mencionado en la parte de vialidad.
\\\\
Por otro lado, en la sección de vialidad, una vez terminadas las estructuras, se procede
a preparar la sub rasante, añadir las sub bases y bases granulares para la recepción de 
la carpeta asfáltica y sus respectivas compactaciones. Luego, se imprima la base, se
aplica el riego de liga y se procede a colocar la capa intermedia de asfalto y 
la carpeta asfáltica de rodadura. Posteriormente, para ayudar como barrera de soporte del 
hormigonado para vereda, se hace la instalación de las soleras tipo A. Para finalizar, se vierte el
hormigón para la construcción de la acera y cuando finalice su curado, se procede a la 
instalación de señaléticas y demarcación requeridas.
\\\\
Las partidas se realizaron en el orden mencionado dado que es lógico y eficiente. Además, sigue un
orden de ejecución que permite que las partidas se realicen de manera continua y sin interrupciones
dentro de lo posible, para que el plazo total de la obra sea el menor posible.

\subsection{Rendimientos por partida}

Los rendimientos por partidas utilizados son los siguientes

\begin{itemize}
    \item \textbf{VIALIDAD}
    \begin{itemize}
        \item \textbf{Preparación área de trabajo}
        \begin{itemize}
            \item Remoción de Pavimento de Hormigón: Se supone un espesor inicial de 0.4 metros, por lo tanto, y considernado un area de 260 $m^2$, 
            el volumen a extraer es de 104 $m^3$. \textbf{El rendimiento utilizado es de 105 $m^3$/día por maquina}. Ademas, es nesesario considerar la remocion de escombros,
            donde se considera un \textbf{rendimiento de 304 $m^3$/día por camion}.
            \item Remoción de Pavimento Asfáltico: Se supone un espesor inicial de 0.1 metros, por lo tanto, y considernado un area de 530 $m^2$,
            el volumena extraer es de 53 $m^3$. \textbf{El rendimiento utilizado es de 105 $m^3$/día por maquina}. Ademas, es nesesario considerar la remocion de escombros,
            donde se considera un \textbf{rendimiento de 304 $m^3$/día por camion}.
        \end{itemize}
        \item \textbf{Movimiento de Tierra}
        \begin{itemize}
            \item Excavación en Terreno de Cualquier Naturaleza (TCN): Se utilizo un \textbf{rendimiento de 105 $m^3$/día por maquina}. demas, es nesesario considerar la remocion de escombros,
            donde se considera un \textbf{rendimiento de 304 $m^3$/día por camion}.
            \item Formación y Construcción de Terraplenes: Se utilizo un \textbf{rendimiento de 105 $m^3$/día por maquina}. demas, es nesesario considerar la remocion de escombros,
            donde se considera un \textbf{rendimiento de 304 $m^3$/día por camion}.
        \end{itemize}
        \item \textbf{Capas Granulares}
        \begin{itemize}
            \item Sub Base Granular CBR 40\%: Para aplicar la capa se utilizo un \textbf{rendimiento de 224 $m^3$/día}. Luego para hacer la compactacion, se utilizo un \textbf{rendimiento de 16800 $m^2$/día}.
            \item Base Granular CBR 80\%: =Para aplicar la capa se utilizo un \textbf{rendimiento de 224 $m^3$/día}. Luego para hacer la compactacion, se utilizo un \textbf{rendimiento de 16800 $m^2$/día}.
        \end{itemize}
        \item \textbf{Pavimentos}
        \begin{itemize}
            \item Imprimación: Se utiliza un \textbf{rendimiento de 3200 $m^2$/día}.
            \item Riego de Liga: Se utiliza un \textbf{rendimiento de 3200 $m^2$/día}.
            \item Concreto Asfáltico, Capa de Rodadura: Se utiliza un \textbf{rendieminto de 2250 $m^2$/día}.
            \item Concreto Asfáltico, Capa Intermedia: Se utiliza un \textbf{rendieminto de 2250 $m^2$/día}.
            \item Aceras de Hormigón: Se utiliza un \textbf{rendieminto de 14 $m^3$/día}.
            \item Soleras Tipo A: Se utiliza un \textbf{rendieminto de 105 $m$/día}.
        \end{itemize}
        \item \textbf{Señalética y Demarcación}
        \begin{itemize}
            \item Señal Vertical Lateral Tipo 2: Se utiliza un \textbf{rendimiento de 10 $UN$/día}
            \item Señal Vertical Lateral Tipo 3: Se utiliza un \textbf{rendimiento de 10 $UN$/día}
        \end{itemize}
    \end{itemize}
    \item \textbf{ESTRUCTURAS}
    \begin{itemize}
        \item Excavación a Máquina en Puente y Estructuras: Se utiliza un \textbf{rendimiento de 910 $m^3$/día por maquina}, considerando la utilización de 2 máquinas retroexcavadoras.
        \item Relleno Estructural Permeable: Se utiliza un \textbf{rendimiento de 1365 $m^3$/día por maquina}, considerando 3 camiones tolva.
        \item Impermeabilización de Muros y Estribos: Se utiliza un \textbf{rendimiento de 240 $m^2$/día}, con 1 cuadrilla de 3 obreros trabajando.
        \item Hormigón G-5: Se utiliza un \textbf{rendimiento de 224 $m^3$/día por maquina}, con 1 cuadrilla de 3 obreros.
        \item Hormigón G-30/SM: Se utiliza un \textbf{rendimiento de 224 $m^3$/día por maquina}, con 1 cuadrilla trabajando.
        \item Acero para Armaduras A63-42H: Se utiliza un \textbf{rendimiento de 3600 $kg$/día}, con 3 cuadrillas trabajando.
        \item Moldajes para Infraestructura: Se utiliza un \textbf{rendimiento de 320 $m^2$/día}, con 1 cuadrilla.
        \item Losa de Acceso: Se utiliza un \textbf{rendimiento de 224 $m^3$/día por maquina}, con 1 cuadrilla.
        \item Barbacanas de desagüe: Se utiliza un \textbf{rendimiento de 24 $UN$/día}, utilizando 1 cuadrilla.
    \end{itemize}
\end{itemize}