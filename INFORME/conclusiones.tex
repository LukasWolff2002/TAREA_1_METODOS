\section{Conclusiones}

En conclusión, esta tarea permite aplicar principios fundamentales de cubicación y programación de obras en un contexto real de infraestructura civil, particularmente en un proyecto de ampliación vial y la construcción de un paso bajo nivel. La correcta estimación de volúmenes y tiempos para actividades como excavaciones, muros de contención y pavimentaciones no solo asegura el cumplimiento de los plazos y costos establecidos, sino que también es crucial para evitar complicaciones técnicas y sobrecostos en la ejecución de la obra.
\\\\
Los objetivos se cumplieron, ya que se realizó la cubicación de las partidas, se obtuvieron los rendimientos de cada una de ellas y se programaron los plazos de ejecución de manera eficiente, respetando un orden de precedencia lógico. La cubicación precisa y la programación adecuada en MS Project, utilizando datos confiables para los rendimientos, permite optimizar los recursos disponibles y planificar la obra de manera eficiente, contribuyendo al éxito del proyecto.
\\\\
De esta manera, la tarea proporciona una visión clara de los desafíos y responsabilidades que conlleva la planificación y ejecución de obras civiles en el mundo real, destacando la importancia de un enfoque técnico y organizado en cada una de sus etapas.